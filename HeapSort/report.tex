\documentclass[UTF8]{ctexart}
\usepackage{geometry, CJKutf8}
\geometry{margin=1.5cm, vmargin={0pt,1cm}}
\setlength{\topmargin}{-1cm}
\setlength{\paperheight}{29.7cm}
\setlength{\textheight}{25.3cm}
\setlength{\headheight}{12.64723pt}


% useful packages.
\usepackage{amsfonts}
\usepackage{amsmath}
\usepackage{amssymb}
\usepackage{amsthm}
\usepackage{enumerate}
\usepackage{graphicx}
\usepackage{multicol}
\usepackage{fancyhdr}
\usepackage{layout}
\usepackage{listings}
\usepackage{float, caption}
\usepackage{xcolor}
\usepackage{tikz}
\usepackage{tikz-qtree}

\lstset{
    basicstyle=\ttfamily, basewidth=0.5em,
    language=C++,
    keywordstyle=\color{blue},
    stringstyle=\color{green},
    commentstyle=\color{gray},
    numbers=left,
    numberstyle=\ttfamily\color{gray}\footnotesize,
    stepnumber=1,
    breaklines=true,
    showstringspaces=false,
    tabsize=4
}

% some common command
\newcommand{\dif}{\mathrm{d}}
\newcommand{\avg}[1]{\left\langle #1 \right\rangle}
\newcommand{\difFrac}[2]{\frac{\dif #1}{\dif #2}}
\newcommand{\pdfFrac}[2]{\frac{\partial #1}{\partial #2}}
\newcommand{\OFL}{\mathrm{OFL}}
\newcommand{\UFL}{\mathrm{UFL}}
\newcommand{\fl}{\mathrm{fl}}
\newcommand{\op}{\odot}
\newcommand{\Eabs}{E_{\mathrm{abs}}}
\newcommand{\Erel}{E_{\mathrm{rel}}}

\begin{document}

\pagestyle{fancy}
\fancyhead{}
\lhead{吴慧阳  3230105143}
\chead{数据结构与算法第七次作业}
\rhead{\today}

\section{\texttt{heapsort}函数的实现思路}
    书中的\texttt{heapsort()}函数大体上分为两步,第一步为\texttt{buildHeap},第二步为\texttt{deleteMax}。
    这两步分别用于将序列构造为最大堆,并将堆顶元素调至序列末尾,同时维持堆的结构。\par
    查询文档可知,在STL库中能实现此两功能的函数分别为\texttt{make\_heap}和\texttt{pop\_heap}。因此,我们只需要调用这两个函数即可实现\texttt{heapsort}。
\section{\texttt{heapsort}函数的具体实现}
    \begin{lstlisting}
    template <typename T>
    void heapSort(std::vector<T>& vec) {
        std::make_heap(vec.begin(), vec.end());  
        for (auto it = vec.end(); it != vec.begin(); --it) {
            std::pop_heap(vec.begin(), it);
        }
    }
    \end{lstlisting}
\section{测试用函数说明及实现}
    为了测试\texttt{heapsort}函数的正确性,我们需要\texttt{check}函数、用于生成测试数据的几个函数与记录时间的函数。
    \texttt{check}函数实现如下:
    \begin{lstlisting}
template <typename T>
bool check(const std::vector<T>& vec) {
    for (size_t i = 1; i < vec.size(); ++i) {
        if (vec[i] < vec[i - 1]) { 
            return false;
        }
    }
    return true;
}
\end{lstlisting}
    至于生成随机数的函数,由于数据范围较大,\texttt{random()}函数无法满足要求,因此
    在查询CSDN后,我们选择使用\texttt{random\_device}函数生成随机数种子,再使用\texttt{mt19937}生成随机数。
    \begin{lstlisting}
std::vector<int> generateRandomSequence(size_t size, int range = 1000000000) {
    std::vector<int> vec(size);
    std::random_device rd;                                 
    std::mt19937 gen(rd());                                
    std::uniform_int_distribution<int> dist(0, range - 1); 
    std::generate(vec.begin(), vec.end(), [&]() {          
        return dist(gen);
    });
    return vec;
}
    \end{lstlisting}
其余测试数据的实现是简单的,此处略去。
我们在实际测试时使用了\texttt{chrono}头文件中的\texttt{high\_resolution\_clock}来记录函数运行前后的时间差,考虑到时间
复杂度,最后结果使用毫秒为单位输出,实现如下:
\begin{lstlisting}
    auto start = std::chrono::high_resolution_clock::now();
        heapSort(vec1);
        auto end = std::chrono::high_resolution_clock::now();
        auto customTime = std::chrono::duration_cast<std::chrono::milliseconds>(end - start).count();
\end{lstlisting}
\section{测试流程}
我们先创建了两个vector,分别命名为\texttt{vec1}和\texttt{vec2},然后分别调用\texttt{generateRandomSequence}函数生成随机数
,然后对其进行了两次测试:一次使用我们创建的\texttt{heapsort},并调用\texttt{check}函数检查排序结果是否正确;
一次使用STL库中的\texttt{heapsort}。最后输出时间。
\section{测试结果}
进行了10次测试后,我们取平均运行时间得到下表:
\begin{table}[h!]
    \centering
    \begin{tabular}{l|r|r}
    \hline
     & my heapsort time & STL heapsort time  \\ \hline
    random sequence & 179.6ms & 173.2ms  \\ \hline
    ordered sequence & 76.0ms & 64.0ms  \\ \hline
    reversed sequence & 84.6ms & 68.4ms  \\ \hline
    repetitive sequence & 158.0ms & 151.2ms  \\ \hline
    \end{tabular}
    \caption{results}
    \end{table}
\section{结论}
    我们发现,我们实现的\texttt{heapsort}函数运行时间略长于STL库中的\texttt{heapsort}函数。
\end{document}

%%% Local Variables: 
%%% mode: latex
%%% TeX-master: t
%%% End: 
