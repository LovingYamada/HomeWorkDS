\documentclass[UTF8]{ctexart}
\usepackage{geometry, CJKutf8}
\geometry{margin=1.5cm, vmargin={0pt,1cm}}
\setlength{\topmargin}{-1cm}
\setlength{\paperheight}{29.7cm}
\setlength{\textheight}{25.3cm}
\setlength{\headheight}{12.64723pt}


% useful packages.
\usepackage{amsmath}
\usepackage{amsfonts}
\usepackage{algorithm}
\usepackage{algorithmic}
\usepackage{graphicx}
\usepackage{ctex}
\usepackage{amsfonts}
\usepackage{amsmath}
\usepackage{amssymb}
\usepackage{amsthm}
\usepackage{enumerate}
\usepackage{graphicx}
\usepackage{multicol}
\usepackage{fancyhdr}
\usepackage{layout}
\usepackage{listings}
\usepackage{float, caption}
\usepackage{xcolor}
\usepackage{tikz}
\usepackage{tikz-qtree}

\lstset{
    basicstyle=\ttfamily, basewidth=0.5em,
    language=C++,
    keywordstyle=\color{blue},
    stringstyle=\color{green},
    commentstyle=\color{gray},
    numbers=left,
    numberstyle=\ttfamily\color{gray}\footnotesize,
    stepnumber=1,
    breaklines=true,
    showstringspaces=false,
    tabsize=4
}

% some common command
\newcommand{\dif}{\mathrm{d}}
\newcommand{\avg}[1]{\left\langle #1 \right\rangle}
\newcommand{\difFrac}[2]{\frac{\dif #1}{\dif #2}}
\newcommand{\pdfFrac}[2]{\frac{\partial #1}{\partial #2}}
\newcommand{\OFL}{\mathrm{OFL}}
\newcommand{\UFL}{\mathrm{UFL}}
\newcommand{\fl}{\mathrm{fl}}
\newcommand{\op}{\odot}
\newcommand{\Eabs}{E_{\mathrm{abs}}}
\newcommand{\Erel}{E_{\mathrm{rel}}}

\begin{document}

\pagestyle{fancy}
\fancyhead{}
\lhead{吴慧阳  3230105143}
\chead{第八次作业}
\rhead{\today}

\section{问题描述}
给定一个整数序列,找到其中的最长严格递增子序列(LIS)的长度。

\section{动态规划算法(时间复杂度 O($n^2$))}
我们可以使用动态规划方法来解决这个问题。定义一个数组 \( dp \),其中 \( dp[i] \) 表示以第 \( i \) 个元素结尾的最长严格递增子序列的长度。
\par 设输入数组为 \( nums \)。
显然状态转移方程为$$    
    dp[i] = max(dp[i], dp[j] + 1) \quad \text{其中} \quad 0 \leq j < i \quad \text{且} \quad nums[j] < nums[i]$$
\subsection{算法描述}
\begin{itemize}
    \item 初始化一个长度为 \( n \) 的数组 \( dp \),使得 \( dp[i] = 1 \)(因为每个元素本身至少能形成一个长度为1的子序列)。
    \item 对于每个索引 \( i \) 从 1 到 \( n-1 \),检查所有比它小的索引 \( j \)(即 \( 0 \leq j < i \)):
    \begin{itemize}
        \item 如果 \( nums[j] < nums[i] \),说明 \( nums[i] \) 可以接在以 \( nums[j] \) 结尾的子序列后面。更新 \( dp[i] = \max(dp[i], dp[j] + 1) \)。
    \end{itemize}
    \item 最终 LIS 的长度是 \( dp \) 数组中的最大值:\( \max(dp) \)。
\end{itemize}

\subsection{O($n^2$) 算法伪代码}
\begin{algorithm}
\caption{最长严格递增子序列(O($n^2$))}
\begin{algorithmic}[1]
\STATE \textbf{输入:} 整数数组 \( nums \)
\STATE \( n \gets \text{nums的长度} \)
\STATE \( dp \gets \text{大小为n的数组,初始化为1} \)
\FOR{ \( i = 1 \text{ 到 } n-1 \)}
    \FOR{ \( j = 0 \text{ 到 } i-1 \)}
        \IF{ \( nums[j] < nums[i] \)}
            \STATE \( dp[i] \gets \max(dp[i], dp[j] + 1) \)
        \ENDIF
    \ENDFOR
\ENDFOR
\STATE \textbf{输出:} \( \max(dp) \)
\end{algorithmic}
\end{algorithm}

\subsection{示例:}
考虑输入序列:\( [10, 22, 9, 33, 21, 50, 41, 60] \)。

\begin{itemize}
    \item 初始化 \( dp = [1, 1, 1, 1, 1, 1, 1, 1] \)。
    \item 对于 \( i = 1 \),更新 \( dp = [1, 2, 1, 1, 1, 1, 1, 1] \) 因为 \( 22 > 10 \)。
    \item 对于 \( i = 2 \),没有更新,\( dp = [1, 2, 1, 1, 1, 1, 1, 1] \)。
    \item 对于 \( i = 3 \),更新 \( dp = [1, 2, 1, 3, 1, 1, 1, 1] \) 因为 \( 33 > 10, 33 > 22 \)。
    \item 按此类推,直到最后。
    \item 最终 \( dp = [1, 2, 1, 3, 2, 4, 4, 5] \),LIS 长度为 5。
\end{itemize}

\section{动态规划加二分法算法(时间复杂度 O($n \log n$))}

我们可以使用二分法对动态规划进行优化,使时间复杂度变为 \( O(n \log n) \)。
优化的关键在于将原动态规划方法的\( dp \)概念进行优化。
我们选择用\( dp[i] \)存储长度为k的递增子序列
的最末元素,若有多个长度为k的递增子序列,则记录
最小的那个。
\subsection{算法描述}
\begin{itemize}
    \item 初始化一个数组 \( dp \),用于存储最小的严格递增子序列结尾元素,\( dp[0] \)为 \(num[0] \)。
    \item 对于每个元素 \( x \):
    \begin{itemize}
        \item 使用二分查找找到 \( dp \) 中第一个大于或等于 \( x \) 的位置。
        \item 如果该位置是数组末尾,说明 \( x \) 可以扩展最长递增子序列;否则,用 \( x \) 替换该位置的元素。
    \end{itemize}
    \item 最终,\( dp \) 数组的长度即为最长严格递增子序列的长度。
\end{itemize}

\subsection{算法伪代码}
\begin{algorithm}
\caption{最长严格递增子序列(O($n \log n$)) 算法}
\begin{algorithmic}[1]
\STATE \textbf{输入:} 整数数组 \( nums \)
\STATE 初始化 \( dp \) 为一个空数组,\( dp[0] \) 为 \( nums[0] \)
\FOR{每个元素 \( x \) 在 \( nums \) 中}
    \STATE 使用二分查找找到 \( dp \) 中第一个大于或等于 \( x \) 的位置,记为 \( pos \)
    \IF{ \( pos == \text{dp的长度} \)}
        \STATE 将 \( x \) 添加到 \( dp \) 的末尾
    \ELSE
        \STATE 将 \( dp[pos] \) 替换为 \( x \)
    \ENDIF
\ENDFOR
\STATE \textbf{输出:} \( \text{dp的长度} \)
\end{algorithmic}
\end{algorithm}
\subsection{示例:}
假设输入序列为:
\([9, 2, 1, 5, 3, 6, 4, 8, 9, 7]\)

执行过程如下:

\begin{itemize}
    \item 初始时,令 \( dp[0] = 9 \), \( len = 1 \)。
    \item 处理 \( num[1] = 2 \),将 \( dp[0] = 2 \), \( len = 1 \)。
    \item 处理 \( num[2] = 1 \),将 \( dp[0] = 1 \), \( len = 1 \)。
    \item 处理 \( num[3] = 5 \),因为 \( 5 > dp[0] \),所以 \( dp[1] = 5 \), \( len = 2 \)。
    \item 处理 \( num[4] = 3 \),使用二分查找找到替换位置,\( dp[1] = 3 \), \( len = 2 \)。
    \item 处理 \( num[5] = 6 \),因为 \( 6 > dp[1] \),所以 \( dp[2] = 6 \), \( len = 3 \)。
    \item 处理 \( num[6] = 4 \),使用二分查找找到替换位置,\( dp[2] = 4 \), \( len = 3 \)。
    \item 处理 \( num[7] = 8 \),因为 \( 8 > dp[2] \),所以 \( dp[3] = 8 \), \( len = 4 \)。
    \item 处理 \( num[8] = 9 \),因为 \( 9 > dp[3] \),所以 \( dp[4] = 9 \), \( len = 5 \)。
    \item 处理 \( num[9] = 7 \),使用二分查找找到替换位置,\( dp[3] = 7 \), \( len = 5 \)。
\end{itemize}

最终,最长递增子序列的长度为 5。

\section{总结}
本文提供了两种解决最长严格递增子序列(LIS)问题的算法。第一种算法使用动态规划,时间复杂度为 \( O(n^2) \),第二种算法使用二分查找,优化了时间复杂度为 \( O(n \log n) \)。

\end{document}

%%% Local Variables: 
%%% mode: latex
%%% TeX-master: t
%%% End: 
