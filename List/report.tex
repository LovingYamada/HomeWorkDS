\documentclass[UTF8]{ctexart}
\usepackage{geometry, CJKutf8}
\geometry{margin=1.5cm, vmargin={0pt,1cm}}
\setlength{\topmargin}{-1cm}
\setlength{\paperheight}{29.7cm}
\setlength{\textheight}{25.3cm}
\setlength{\headheight}{12.64723pt}


% useful packages.
\usepackage{amsfonts}
\usepackage{amsmath}
\usepackage{amssymb}
\usepackage{amsthm}
\usepackage{enumerate}
\usepackage{graphicx}
\usepackage{multicol}
\usepackage{fancyhdr}
\usepackage{layout}
\usepackage{listings}
\usepackage{float, caption}
\usepackage{xcolor}

\lstset{
    basicstyle=\ttfamily, basewidth=0.5em,
    language=C++,
    keywordstyle=\color{blue},
    stringstyle=\color{green},
    commentstyle=\color{gray},
    numbers=left,
    numberstyle=\ttfamily\color{gray}\footnotesize,
    stepnumber=1,
    breaklines=true,
    showstringspaces=false,
    tabsize=4
}

% some common command
\newcommand{\dif}{\mathrm{d}}
\newcommand{\avg}[1]{\left\langle #1 \right\rangle}
\newcommand{\difFrac}[2]{\frac{\dif #1}{\dif #2}}
\newcommand{\pdfFrac}[2]{\frac{\partial #1}{\partial #2}}
\newcommand{\OFL}{\mathrm{OFL}}
\newcommand{\UFL}{\mathrm{UFL}}
\newcommand{\fl}{\mathrm{fl}}
\newcommand{\op}{\odot}
\newcommand{\Eabs}{E_{\mathrm{abs}}}
\newcommand{\Erel}{E_{\mathrm{rel}}}

\begin{document}

\pagestyle{fancy}
\fancyhead{}
\lhead{吴慧阳  3230105143}
\chead{数据结构与算法第四次作业}
\rhead{\today}

\section{测试程序的设计思路}
在完成了\texttt{--}的设计后,我将\texttt{List.h}中的功能分为了如下几大类:\par
\begin{itemize}
    \item 创造消灭列表类\begin{itemize}
        \item \texttt{List()}
        \item \texttt{List(std::initializer\_list<Object> il)}
        \item \texttt{List(const List \&rhs)}
        \item \texttt{List \&operator=(List copy)}
        \item \texttt{List(List \&\&rhs)}
        \item \texttt{$\sim$List()}
    \end{itemize}
    \item 操作列表类\begin{itemize}
        \item \texttt{push}
        \item \texttt{pop}
        \item \texttt{insert}
        \item \texttt{erase}
        \item \texttt{clear}
    \end{itemize}
    \item 输出列表及其特征类\begin{itemize}
        \item \texttt{display}
        \item \texttt{front}
        \item \texttt{back}
        \item \texttt{end}
        \item \texttt{begin}
        \item \texttt{size}
        \item \texttt{empty}
    \end{itemize}
\end{itemize}
其中,\texttt{display}是我为了更方便地检测功能而设计的一个函数,其功能
是将\texttt{list\_}的全部数据顺序输出.
其具体实现如下:\begin{lstlisting}
    void display() const
    {
        if(empty())
        {
            std::cout << "Empty List." << std::endl;
        }
        else
        {
            for(auto it = begin() ; it != end() ; ++it)
            {
                std::cout << *it << " ";
            }
            std::cout << std::endl;
        }
    }
\end{lstlisting}
于是我使用了如下程序来测试这些函数:
\begin{lstlisting}
    
#include "List.h"

int main()
{
    List<int> intList;//constructor
    
    if(intList.empty()) std::cout << "Empty" << std::endl;
    intList.push_back(520);
    std::cout << "List size : " << intList.size() << std::endl;
    if(!intList.empty()) std::cout << "Nonempty" << std::endl;
    intList.push_back(1314);
    std::cout << "List size : " << intList.size() << std::endl;
    intList.display();
    intList.push_front(1234);
    intList.display();
    int rval = 10;//push
    intList.push_back(std::move(rval));
    intList.push_front(std::move(rval));
    intList.display();
    intList.pop_back();//pop
    intList.pop_front();
    intList.display();

    //insert begin end
    auto it = intList.begin();
    ++it;
    auto it_2 = intList.insert(it, 15);
    intList.display();
    std::cout << "The node whose position coincides what insert returns has value : " << 
    *it_2 << std::endl;
    std::cout << "The iterator it now points to " << *it << std::endl;
    it_2 = intList.end();
    --it_2;
    std::cout << "The iterator it_2 now points to " << *it_2 << std::endl;
    //const begin end
    List<int>::const_iterator it_3 = intList.begin();
    std::cout << "The iterator it_3 points to " << *it_3 << std::endl;
    it_3 = intList.end();
    it_3--;
    std::cout << "The iterator it_3 points to " << *it_3 << std::endl;
    //rvalue
    it_2 = intList.insert(it, std::move(rval));
    intList.display(); 
    std::cout << "The node whose position coincides what insert returns has value : " << 
    *it_2 << std::endl;
    std::cout << "The iterator it now points to " << *it << std::endl;

    //erase
    intList.erase(it);
    intList.display();
    it++;//++
    it_2--;
    --it_2;//--
    std::cout << "it_2 now points to : " << *it_2 << std::endl
    <<"it now points to : " << *it << std::endl;
    intList.erase(it_2, it);
    intList.display();
    intList.push_front(520);
    intList.clear();//clear
    intList.display();
    //constructor
    std::cout << "intializer list : "; List<int>intList_1{1,2,3,4,5,6};
    intList_1.display();
    List<int>intList_2(intList_1);
    std::cout << "copy : "; intList_2.display();
    List<int>intList_3(std::move(intList_1));
    std::cout << "move : "; intList_3.display();
    intList_1.display();//move
    //operator
    intList = (intList_2);
    intList.display();
    //front back
    std::cout << "Front : " << intList.front() << std::endl;
    std::cout << "Back : " << intList.back() << std::endl;
    const List<int> intList_const = intList;
    intList_const.display();
    std::cout << "Const front : " << intList_const.front() << std::endl;
    std::cout << "Const back : " << intList_const.back() << std::endl;
}
\end{lstlisting}
这个程序首先利用\texttt{List()}创建了\texttt{intList}
并用\texttt{empty}
检查其是否是空的\par
之后我在\texttt{intList}中检验了各类
\texttt{push}和\texttt{pop}的功能,
同时还判断了\texttt{empty}在\texttt{intList}非空
的时候是否能正确运行\par
在完成了\texttt{push pop empty}的测试后,
我开始测试\texttt{insert erase begin end}等与\texttt{iterator}有关的功能能否正常实现.\par
在之后,我则对除默认构造函数外的构造函数进行了相应的测试.
分别测试了以\texttt{initializer\_list}为参数的构造函数和拷贝构造函数
、移动构造函数是否能正常初始化\texttt{List}.\par
在剩余的代码中,我简单的对赋值运算符、\texttt{front}与\texttt{back}进行了相应的测试,
结果均符合预期.
\section{测试的结果}

测试结果如下:\begin{lstlisting}
Empty
List size : 1
Nonempty
List size : 2
520 1314 
1234 520 1314 
10 1234 520 1314 10 
1234 520 1314 
1234 15 520 1314 
The node whose position coincides what insert returns has value : 15
The iterator it now points to 520
The iterator it_2 now points to 1314
The iterator it_3 points to 1234
The iterator it_3 points to 1314
1234 15 10 520 1314 
The node whose position coincides what insert returns has value : 10
The iterator it now points to 520
1234 15 10 1314 
it_2 now points to : 1234
it now points to : 1314
1314 
Empty List.
intializer list : 1 2 3 4 5 6 
copy : 1 2 3 4 5 6 
move : 1 2 3 4 5 6 
Empty List.
1 2 3 4 5 6 
Front : 1
Back : 6
1 2 3 4 5 6 
Const front : 1
Const back : 6
\end{lstlisting}

我用 valgrind 进行测试,发现没有发生内存泄露。

\end{document}

%%% Local Variables: 
%%% mode: latex
%%% TeX-master: t
%%% End: 
