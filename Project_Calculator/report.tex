\documentclass[UTF8]{ctexart}
\usepackage{geometry, CJKutf8}
\geometry{margin=1.5cm, vmargin={0pt,1cm}}
\setlength{\topmargin}{-1cm}
\setlength{\paperheight}{29.7cm}
\setlength{\textheight}{25.3cm}
\setlength{\headheight}{12.64723pt}


% useful packages.
\usepackage{amsfonts}
\usepackage{amsmath}
\usepackage{amssymb}
\usepackage{amsthm}
\usepackage{enumerate}
\usepackage{graphicx}
\usepackage{multicol}
\usepackage{fancyhdr}
\usepackage{layout}
\usepackage{listings}
\usepackage{float, caption}
\usepackage{xcolor}
\usepackage{tikz}
\usepackage{tikz-qtree}

\lstset{
    basicstyle=\ttfamily, basewidth=0.5em,
    language=C++,
    keywordstyle=\color{blue},
    stringstyle=\color{green},
    commentstyle=\color{gray},
    numbers=left,
    numberstyle=\ttfamily\color{gray}\footnotesize,
    stepnumber=1,
    breaklines=true,
    showstringspaces=false,
    tabsize=4
}

% some common command
\newcommand{\dif}{\mathrm{d}}
\newcommand{\avg}[1]{\left\langle #1 \right\rangle}
\newcommand{\difFrac}[2]{\frac{\dif #1}{\dif #2}}
\newcommand{\pdfFrac}[2]{\frac{\partial #1}{\partial #2}}
\newcommand{\OFL}{\mathrm{OFL}}
\newcommand{\UFL}{\mathrm{UFL}}
\newcommand{\fl}{\mathrm{fl}}
\newcommand{\op}{\odot}
\newcommand{\Eabs}{E_{\mathrm{abs}}}
\newcommand{\Erel}{E_{\mathrm{rel}}}

\begin{document}

\pagestyle{fancy}
\fancyhead{}
\lhead{吴慧阳  3230105143}
\chead{数据结构与算法项目作业 计算器}
\rhead{\today}

\section{设计思路}
\subsection{基本思路}
\begin{itemize}
    \item 使用栈来存储数字和运算符。
    \item 遇到数字时,将其压入数字栈。
    \item 遇到运算符时:
    \begin{itemize}
        \item 如果运算符栈为空,或者运算符栈顶为左括号,则直接压入运算符栈。
        \item 如果运算符栈顶为右括号,则将运算符栈中的运算符依次弹出并计算,直到遇到左括号。
        \item 若为其他运算符,则比较当前运算符与栈顶运算符的优先级。若当前优先级较低,则先弹出栈顶运算符并计算,再将当前运算符压入。
    \end{itemize}
    \item 遍历表达式结束后,将栈中剩余的运算符依次弹出并计算。
\end{itemize}

\subsection{优先级规则}
\begin{itemize}
    \item 运算符优先级从高到低依次为:括号、乘除、加减。
    \item 左括号的优先级最低,用于标记优先计算范围。
\end{itemize}

\section{程序实现}
\subsection{数据结构设计}
\begin{itemize}
    \item \textbf{数字栈:} 用于存储操作数。
    \item \textbf{运算符栈:} 用于存储运算符和括号。
\end{itemize}

\subsection{算法步骤}
\begin{enumerate}
    \item 初始化两个栈:数字栈和运算符栈。
    \item 从左到右扫描表达式中的字符:
    \begin{itemize}
        \item 如果是数字,将其压入数字栈。
        \item 如果是左括号或运算符,将其压入运算符栈。
        \item 如果是右括号:
        \begin{itemize}
            \item 弹出运算符栈顶的运算符,并从数字栈弹出对应的操作数进行计算。
            \item 将计算结果压入数字栈,直到遇到左括号。
        \end{itemize}
        \item 如果是运算符:
        \begin{itemize}
            \item 比较当前运算符与栈顶运算符的优先级。
            \item 若当前运算符优先级较低,则弹出栈顶运算符并计算,重复此过程,直至满足条件后将当前运算符压入。
        \end{itemize}
    \end{itemize}
    \item 扫描完成后,依次弹出运算符栈中的运算符进行计算,直到栈为空。
\end{enumerate}
\subsection{字符串转数字}
解析字符串为数字是表达式求值的核心操作之一。
C++ 标准库中的 \texttt{std::stod} 函数提供了高效且可靠的解决方案,其特点如下:

\begin{enumerate}
    \item \textbf{功能全面}:\texttt{std::stod} 支持解析常见的数字格式,包括:
    \begin{itemize}
        \item 整数和小数(例如 \texttt{"123"} 或 \texttt{"123.45"})。
        \item 负数(例如 \texttt{"-123"} 或 \texttt{"-123.45"})。
        \item 科学计数法(例如 \texttt{"1.23e4"})。
    \end{itemize}
    \item \textbf{错误处理}:\texttt{std::stod} 可以处理并报告解析过程中可能出现的错误,例如:
    \begin{itemize}
        \item 非法字符(例如 \texttt{"abc"})。
        \item 无效的数字格式(例如 \texttt{"123.45.67"})。
\end{enumerate}

\subsection{代码实现}
\begin{lstlisting}
    
\end{lstlisting}
\section{测试与结果}
\subsection{测试用例}
\begin{itemize}
    \item 测试表达式1:$3 + 5 \times 2 - ( 6 / 3 )$
    \item 测试表达式2:$( 2 + 3 ) \times ( 5 - 2 )$
    \item 测试表达式3:$4 \times ( 6 + 2 ) / 8$
\end{itemize}

\subsection{运行结果}
\begin{itemize}
    \item 表达式1结果:$10$
    \item 表达式2结果:$15$
    \item 表达式3结果:$4$
\end{itemize}


\end{document}

%%% Local Variables: 
%%% mode: latex
%%% TeX-master: t
%%% End: 
